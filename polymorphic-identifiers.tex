%-----------------------------------------------------------------------------
%
%    Domain Specific Identifiers
%
%-----------------------------------------------------------------------------


\documentclass[preprint,authoryear,10pt]{sigplanconf}


\usepackage{amsmath}
\usepackage{epigraph}
\usepackage[colorlinks=true,
        linkcolor=black,
        citecolor=black,
        filecolor=black,
        pagecolor=black,
        urlcolor=black,
        bookmarks=true,
        bookmarksopen=true,
        bookmarksopenlevel=3,
        plainpages=false,
        pdfpagelabels=true]{hyperref}

\begin{document}

%\conferenceinfo{WXYZ '05}{date, City.} 
%\copyrightyear{2011} 
%\copyrightdata{[to be supplied]} 

%\titlebanner{banner above paper title}        % These are ignored unless
%\preprintfooter{short description of paper}   % 'preprint' option specified.

\title{Polymorphic Identifiers}
%\subtitle{Subtitle Text, if any}

\authorinfo{Marcel Weiher}
           {metaobject ltd.}
           {marcel@metaobject.com}
\authorinfo{Robert Hirschfled}
           {Hasso Plattner Institut}
           {hirschfeld@acm.org}

\maketitle

\begin{abstract}
We introduce polymorphic identifiers, a generalization of the monomorphic identifiers found in most
programming languages today as first class objects, as well as mechanisms for
resolving those identifiers and binding them into a general purpose programming language.
These identifiers allow stratified designs clearly separating meta- and base-level constructs
as well as light-weight domain-specificity with high levels of expressiveness, 
\end{abstract}

\category{CR-number}{subcategory}{third-level}

\terms 
\keywords
\setlength{\epigraphrule}{0pt}


\section{Introduction}
\epigraph{There are only two hard things in Computer Science:  cache invalidation and naming things} {Phil Karlton}

This is not a paper about cache invalidation.  Mostly.   Rather, we report on an experimental
technique for expanding our ability to directly name many different things in 
programming languages using a common and extensible identifier and identifier resolution
mechanism.
These polymorphic identifiers are not just user-extensible,
but can also be manipulated in the programming language, making them first class 
entities.

Our starting point is a programming model such as that of Smalltalk, where we have objects
that communicate via messages\cite{Goldberg1983}.   Identifiers are used to identify 
objects, but are early bound by the compiler and resolved to slots in local contexts, and 
global, instance-, class- or pool-variables.  Late-binding of names, encapsulation and
polymorphic behavior can only be achieved with message sends.

We then give various examples both of additional things that one might want to reference 
 apart from objects resident in a single address space  and
desirable operations on identifiers beyond resolving them to objects at compile time.

We believe that the accidental complexity \cite{Brooks87nosilver} introduced by the
plethora of ad-hoc mechanisms for addressing these desiderata can be reduced 
by unifying them using the polymorphic identifiers introduced in this paper.


\section{Identifiers}
\label{identifiers}
\epigraph{When programming a component, the right computation model for the component is the least expressive model that results in a natural program.}{Rule of Least Expressiveness.}

Identifiers are symbols that serve as human-readable names for entities/objects
in a program.  They are converted to actual references, usually with a mixture of
compile-time and run-time lookups.  For example, to obtain the value of the local
variable named {\bf anObject}, one would have to search the local context for
a slot named {\bf 'anObject'} and then return the value stored at that slot.


\begin{figure}[htbp]
\begin{center}
\begin{verbatim}
(localContext variableNamed:'�anObject�') value.
\end{verbatim}
\caption{Lookup of a local variable}
\label{variable-lookup}
\end{center}
\end{figure}


Some  languages actually perform all these steps at runtime, but most 
have a compiler help with efficiency by converting the names to indexes into
the local context and the value lookup into a simple load from memory.

Most programming languages don't just optimize the implementation of the identifier
lookup operation, but also the interface.  Instead of having to specify how to
look up the identifier in the local context, the programmer can just write the
identifier {\bf anObject}, which the programming language will treat as equivalent to the
value that will be obtained by looking up the name in the local context.  


\begin{figure}[htbp]
\begin{center}
\begin{verbatim}
	anObject.
\end{verbatim}
\caption{Identifer-oblivious variable lookup}
\label{plain-identifier}
\end{center}
\end{figure}

This  {\em identifier-oblivious} style shown in Figure~\ref{plain-identifier}, where lookup of the value bound to and
identifier is implied, is so familiar that we don't usually think about it, we just
refer to object using the identifier.
  The 
alternative {\em identifier-aware} style shown in Figure~\ref{variable-lookup} would get cumbersome quickly.

Non-local entities such as sub-parts of constructed objects 
cannot be referred to symbolically by a single identifier.
Instead, they must be obtained using computation, a 
sequence of operations, which in Smalltalk is performed using message sends.
For example, assuming we have a model of a robot, we might obtain the
color of the robot's left eye as follows:

\begin{figure}[htbp]
\begin{center}
\begin{verbatim}
	robot head leftEye color.
\end{verbatim}
\caption{Obtaining a reference via chained message sends}
\label{non-local-reference}
\end{center}
\end{figure}


Although we can hide an object reference behind a computational interface, we cannot
hide a computation behind a object reference interface.
Thus any kind of external API has to be presented through a computational, messaging 
interface, even if the intention is not computation, but just access to a resource,
  and all information hiding also requires messaging.  This has led languages
like Self \cite{UngarS87} and Newspeak \cite{newspeak} to effectively abandon identifiers as symbolic names for objects, 
all objects are instead obtained by sending messages to other objects.

On the World Wide Web with its REST architectural model the situation is exactly the reverse: 
 the interface
to resources is their identifier, the Uniform Resource Locator (URL).   Wether
computation happens in providing the resource is a hidden implementation detail
(ignoring the messaging protocols usually required to transmit the resource
to the client).  


\begin{figure}[htbp]
\begin{center}
\begin{verbatim}
http://www.shipper.com/track/12345678
\end{verbatim}
\caption{A Uniform Resource Locator}
\label{url-example}
\end{center}
\end{figure}

So instead of having
a package-tracking application that takes a parameter (2 entities),
one can simply take the view that ``every package has its own 
web-site'', which is then just one entity referenced by a single,
parameterless identifier.  Instead of hiding references behind messaging,
messaging is hidden behind an identifier, and RESTful web applications
perform computations by dereferencing a series of identifiers, that 
are provided incrementally and dynamically in responses.

\begin{figure}[htbp]
\begin{center}
\begin{verbatim}
urn:isbn:123123123
\end{verbatim}
\caption{URN}
\label{urn-example}
\end{center}
\end{figure}

Whereas URLs specify an access path, including a network protocol, Uniform 
Resource Names like the ISBN in Figure~\ref{urn-example} just specify a logical
name, leaving the resolution of that name completely open.  In a sense, URNs
correspond to interfaces 

Uniform Resource Identifiers are a generalization of both Uniform Resource Locators
and Uniform Resource Names.  

URIs are a generalization of Uniform Resource Locators, which specify a
resource by its network location and access path and Uniform Resource Names,
which specify
a resource by a logical name and leave it to system processes to 
determine the exact location.  In practice, most URIs are located on
a continuum with URNs on one end and URLs on the other.

\section{User-level identifiers}


In addition to the entities that can be referred to directly by the programming language's 
built-in identifiers, there are many other entities and situations where that is not possible.
However, entities still need to be addressed, so developers are required
to construct their own identifiers or references and methods for resolving them to the actual
entities.

\subsection{External Resources}

Files, for example are referenced by their filename, or more precisely
a string that represents their access path in the filesystem from the root 
directory.   The operating system is responsible for mapping these full
paths to entities allowing access to the contents of the file.  For example accesing
the current user's ``.bashrc'' file looks as follow:

\begin{figure}[htbp]
\begin{center}
\begin{verbatim}
   char fullname[MAXPATHLEN];
   snprintf(fullname,MAXPATHLEN,"%s/.bashrc",
                            getenv("HOME"));
   int fd = open( filename );
   if ( fd >= 0 ) { 
      //... read contents of file using read()
      close(fd);
   }
\end{verbatim}
\caption{Resolving a name in the user's home directory to a file}
\label{posix-file-resolve}
\end{center}
\end{figure}

The string represents a user-level identifier, which is just a plain character
string that has no meaning to the programming language and must be 
manually resolved by the developer in the {\em identifier-aware} style 
introduced in Section~\ref{identifiers}.  In fact, two different identifiers
are resolved in Figure~\ref{posix-file-resolve}:  first, the identifier {\bf HOME}
from the domain of environment variables is needed, which requires calling
the {\bf getenv()} function to resolve the identifier string to a value.  That result,
also a string, is then combined with the name of the file in question and passed
to the {\bf open()} system call to return an actual file reference.  Boundary conditions
and error checking have largely been elided.

Convenience APIs can ameliorate the situation somewhat, for example shortening 
the multiple steps above into a single (slightly verbose) expression and making the
string processing safer.  However, they cannot
change the fundamental fact that the identifiers in question are basic strings,
manipulating them requires string processing and user-level resolution of the
string to an entity.

\begin{figure}[htbp]
\begin{center}
\begin{verbatim}
   NSData dataWithContentsOfFile:
     ((NSProcessInfo processInfo
        environment 
        objectForKey:'HOME') 
       stringByAppendingPathComponent:'.bashrc')

\end{verbatim}
\caption{Cocoa convenience API for accessing file content}
\label{cocoa-file-contents}
\end{center}
\end{figure}


Which identifiers we can use in an oblivious style and which we need to
be aware of and resolve ourselves is pre-determined by the programming
language.
The same identifier that needs to be created using string processing from various
components in both the POSIX and the Cocoa can be written directly in a typical
Unix shell, including the variable part.

\begin{figure}[htbp]
\begin{center}
\begin{verbatim}
	cat $HOME/.basrhc
\end{verbatim}
\caption{File contents in a shell}
\label{sh-file-contents}
\end{center}
\end{figure}

The lack of string quotes is not just cosmetic: both the variable and the filename
are not just opaque strings, but entities that the shell is aware of and can identify
using the identifiers provided,for example being able to assist by autocomplete 
or wildcard expansion.



\subsection{Dynamic data structures}

Path-based access is not restricted to external resources, but also
prevalent when accessing dynamic data structures

 such as an XML DOM (xpath)

\begin{itemize}
\item [mpw] more details
\end{itemize}

\subsection{User Interfaces}




\subsection{Network Resources}



\subsection{The Trouble with Strings}

\epigraph{The string is a stark data structure and everywhere it is passed there is much duplication of process. It is a perfect vehicle for hiding information.}{Alan Perlis}


One aspect that virtually all user-level identifier mechanisms have in common is that they
are based on string processing.  The use of strings makes user-level identifiers relatively
close to language identifiers ({\bf \hbox{anObject}} vs {\bf \hbox{'anObject'}}), but brings with it a number of problems in terms of overhead, correctness and safety.

As we saw above, the first problem is that a string is not an identifier, and therefore
must be passed as an argument to some function or method to actually return
 an object, either doubling (function+
identifier) or tripling (class/object + message + string ) the number of entities involved to 
retrieve the desired entity.  

The compiler does not have insight into strings that may be used as user-level 
identifiers at some point in the future, and therefore cannot determine wether 
a string-based identifier is correct, or even syntactically well-formed.   The compiler
and runtime also cannot help with turning the identifier into a more efficient representation
like an index into a local store.  Instead, lookup must be performed using hash-tables
or other dictionary structures at runtime on individual path components, 
and complex paths must be first be decomposed into their components.

All this run-time processing of strings is performed with general-purpose string
processing libraries that do not take the identifier semantics into account, with
many string operations capable of turning valid identifiers into invalid ones.
While it was fairly easy to paper over the differences between strings as 
data structures for human-readable text and strings as identifiers for programs,
the differences are becoming more obvious with internationalization, as the
problems making Perl UTF8 or Unicode aware demonstrate \cite{perl-unicode}

Mixing strings-as-use-data with strings-as-identifiers also has well known
security problems such as shell-escaping and SQL-injection, one of the
biggest sources of vulnerabilities on the web.



\begin{figure}[htbp]
\begin{center}
\begin{verbatim}
u := NSURL alloc
           initWithScheme:'http'
           host:'www.example.org'
           path:'/' 
\end{verbatim}
\caption{URL initialized in an object-oriented fashion}
\label{url-as-obj}
\end{center}
\end{figure}

On the other hand, using non-string types, while increasing safety somewhat,
obscures intent further by breaking up the identifiers.
It is certainly debatable wether the more object-oriented
initialization of Figure~\ref{url-as-obj}  has any practical benefits over a
simple initialization from string as in Figure~\ref{url-as-string}, and in
the end the components are still strings and must be converted/combined
into the full URL at runtime.


\begin{figure}[htbp]
\begin{center}
\begin{verbatim}
u := NSURL URLWithString:'http://www.example.org'
\end{verbatim}
\caption{URL initialized with a single string}
\label{url-as-string}
\end{center}
\end{figure}





\section{Polymorphic Identifiers}
\epigraph{A change in perspective is worth 80 IQ points}{Alan Kay}

Syntactically, Polymorphic Identifiers are URIs \cite{rfc3986}, so external resources
such as web pages or files can be addressed directly in an identifier-oblivious
style.  The examples in Figure~\ref{first-polymorphic-examples} show a plain
identifier that will retrieve the ACM's home page, an assignment that will read
the contents of a file into a local variable and finally an assignment that will
write the contents of the local variable to a file, effecting a file copy operation.


\begin{figure}[htbp]
\begin{center}
\begin{verbatim}
http://www.acm.org/
joesBashrc := file:/Users/joe/.bashrc
file:/Users/mike/.bashrc := joesBashrc
\end{verbatim}
\caption{Polymorphic Identifiers in simple expressions}
\label{first-polymorphic-examples}
\end{center}
\end{figure}

\subsection{Schemes}

The scheme name part of the identifier symbolically designates the handler object which 
is used to resolve the remainder of the identifier.   A user-extensible class hierarchy of
scheme resolvers handle 
return bindings that map to actual objects. The scheme resolvers are made available to the 
language by associating them with a scheme name in the special {\bf scheme} scheme,
as shown in Figure~\ref{scheme-scheme-http}. 

\begin{figure}[htbp]
\begin{center}
\begin{verbatim}
scheme:http := URLSchemeResolver scheme.
scheme:file := FileSchemeResolver scheme.
\end{verbatim}
\caption{Adding the http scheme}
\label{scheme-scheme-http}
\end{center}
\end{figure}

The scheme scheme can also be queried,
so the expression {\bf scheme:http} returns the currently installed http handler, and 
{\bf scheme:scheme} returns the scheme scheme, allowing the installed schemes to
be listed.

\begin{figure}[htbp]
\begin{center}
\begin{verbatim}
scheme:scheme
    file, ftp, default, var, bundle, defaults, 
    sel, mainbundle, app, scheme, http,
    env, ref, class,  https
\end{verbatim}
\caption{List of schemes via scheme:scheme}
\label{scheme-scheme}
\end{center}
\end{figure}

\subsection{In-memory access}

The {\bf var} scheme refers to in-memory variables, so {\bf var:anObject} references the
local variable {\bf anObject}.   Compound paths can be given and are resolved in
cooperation with the object, usually by sending unary messages.

The {\bf default} scheme 
is a special scheme that is substituted when a scheme is not otherwise provided.
With the {\bf default} scheme set to the {\bf var} scheme, the {\bf var:} can be 
dropped and we can refer to the local {\bf anObject} variable just by its name.
However, the default scheme can be set to some other scheme if that is more
convenient for the task at hand, as shown in Figure~\ref{file-as-default-scheme},
where a file copy is performed after the default scheme is set to file.

\begin{figure}[htbp]
\begin{center}
\begin{verbatim}
scheme:default := scheme:file
myBashRc := .basrhc
\end{verbatim}
\caption{File copy with file: as default scheme}
\label{file-as-default-scheme}
\end{center}
\end{figure}

One syntactic limitation of the default scheme mechanism is that it does not allow
compound paths, so for compound paths a full URI with scheme must be used.


\subsection{First class references}

The bindings returned by scheme-resolvers are usually ephemeral, used just 
to get to the value pointed to by the identifier, either reading or writing it.  This
makes it possible to program with arbitrary and user-defined identifiers in
an identifier-oblivious style.  It also means that for sufficiently simple scheme-resolvers,
the binding can actually be elided if the compiler can determine that it is not needed.

However, as we saw earlier, it is sometimes
desirable to use the identifier-aware style.  In order to do this, the language
must provide access to the bindings used in resolving the identifier.

The {\bf ref} scheme prevents evaluation of the binding and thus allows access
to the binding itself.  

\begin{figure}[htbp]
\begin{center}
\begin{verbatim}
a := 42
b := ref:var:a.
b value  -> 42
b bindValue:2
a -> 2
\end{verbatim}
\caption{ref}
\label{ref-binding}
\end{center}
\end{figure}


\subsection{Abstraction}



\begin{figure}[htbp]
\begin{center}
\begin{verbatim}
file:{env:HOME}/.bashrc
\end{verbatim}
\caption{User's .bashrc via parametrized polymorphic identifiers}
\label{bashrc-pi}
\end{center}
\end{figure}


\begin{figure}[htbp]
\begin{center}
\begin{verbatim}
base := ref:http://datatracker.ietf.org/doc/.
scheme:rfc := RelScheme schmeWithBase:base.
uriSyntax := rfc:rfc2396.
\end{verbatim}
\caption{A customized scheme for looking up RFCs}
\label{rfc-scheme}
\end{center}
\end{figure}





\begin{itemize}
\item be able to extend the language  / the types of objects 
\item encapsulate processing required for user-extensible identifiers / resource access
\item oblivious or aware not dependent on type of reference
\item runtime or compile-time not dependent on type of reference
\item computational or referential interface at will 
\item ability to 

\end{itemize}



\begin{itemize}


\item Semantics-preserving operations on identifiers \{\} , relative schemes, composite schemes

\end{itemize}


\section{Examples}

\begin{figure}[htbp]
\begin{center}
\begin{verbatim}
 host := 'citeseerx.ist.psu.edu'.
 viewCiteSeer:=ref:http://{host}/viewdoc/summary
 viewCiteSeer getWithArgs doi:'10.1.1.41.7628'
\end{verbatim}
\caption{URL arguments via reference and higher order message}
\label{url-args}
\end{center}
\end{figure}




\begin{figure}[htbp]
\begin{center}
\begin{verbatim}
tell application "iTunes"
  set trackname to name of current track
end tell
tell application "iChat"
  set status message to trackname
end tell
\end{verbatim}
\caption{Using AppleScript to set chat status from track name}
\label{AppleScript}
\end{center}
\end{figure}


\begin{figure}[htbp]
\begin{center}
\begin{verbatim}
app:iChat/statusMessage := app:iTunes/currentTrack/name
\end{verbatim}
\caption{Access to applications via polymorphic identifiers}
\label{NonAppleScript}
\end{center}
\end{figure}



\section{Implementation}



\section{Related Work}

\subsection{LISP Quoting}

LISP quoting allows references to not be evaluated.

\subsection{E}

The E language\cite{MillerRobustComposition}  adds URL for far references.

\subsection{Domain Specific Languages}

\subsection{Key Value Coding, Observing, Binding}

\subsection{ThingLab}

ThingLab adds path-based access to the base Smalltalk language in order to support
constraints referencing parts of objects, say the vertex of a triangle.   It does not
expand this path-based access to other domains or make it user-extensible.

\subsection{Path-based and indirect access in programming languages}

Programming languages such as C, C++ or Pascal have facilities for composite
identifiers, for example using the dot or {\bf -> } operators, but these facilities allow
direct access to the internal data, as does handing out pointers to internal parts.

Properties in C\# and Objective-C 2.0 allow a variable-access interface to be backed by 
message sending, thus allowing path-based access under dynamic control of the object
in question.  However, this does not allow user-defined identifiers or 

\subsection{Icon}

In the Icon programming language \cite{IconRef} , identifiers (references) can be first class values,
and there are complex rules for when a reference remains a reference and when
it's value is used.  However, the rules are implicit and there is no way for the 
developer to choose when to be identifier-aware or identifier-oblivious.  Furthermore,
there is no way for the user to extend the language with new types of references 
or behaviors.



\subsection{Proxies, Composition Filters etc.}

Proxies (Miller paper), Foote/Johnson, various future/lazy systems, Distributed Smalltalk,
NeXT/Apple Distributed Objects.  Issues:  ``Note on Distributed Computing'', treating
remote objects as if they were local and distribution didn't exist.  

Our approach differs in that we handle the difference between remote and local at
the identifier/reference level, rather than at the object (distributed objects) or 
message/call level (RPC).  We would argue that the proxies used in distributed
objects can be regarded as first-class identifiers in our sense, and that the 
metaprogramming techniques that are typically employed are there to go from
the identifier-aware style to an identifier-oblivious style.  However, doing this
in the ``object space'' mixes user-level and system-level functionality, with
for example the typical difficulty of controlling the proxy itself when all messages
are forwarded to the remote object.

\subsection{Internet Programming Languages and Systems}

Wheat \cite{wheat}, Web-Machine, Presto, Resource Oriented Programming,
various specialized systems that export a network file-system.

\subsection{User Level File Systems}

User level filesystems like FUSE   \cite{fuse} or the BSD  ? \cite{bsdfuse} allow
user-level code to add new filesystems to the Unix kernel, without requiring
kernel modifications or impacting operating system kernel stability.

This approach has enabled many types of remote resources such as 
YouTube videos with atttributes and thumbnails or local resources
such as images with their attributes to be accessed using common
POSIX file APIs or convenience APIs built on top of them.

While such integration is a

However:  OS-level, so mediated via the kernel, so cannot reasonably incorporate
local, in-memory resources.   Not integrated into programming language.

But:  should be possible to bridge both-ways, would be very cool \ldots


\section{Summary and Outlook}

Introduced user-extensible identifiers based on URL syntax and 
integrated them as a fundamental part of a programming language
as the sole identifier lookup/evaluation system.



%\appendix
%\section{Appendix A}

%This is the text of the appendix, if you need one.

%\acks

%Acknowledgments, if needed.

% We recommend abbrvnat bibliography style.

\bibliographystyle{abbrvnat}

% The bibliography should be embedded for final submission.
\bibliography{polymorphic-identifiers}


\end{document}
